    \documentclass[12pt,a4paper]{article}
\usepackage[utf8]{inputenc}
\usepackage[french]{babel}
\usepackage{geometry}
\usepackage{tikz}
\usepackage{xcolor}
\usepackage{amsmath,amsfonts,amssymb}
\usepackage{graphicx,float,hyperref,array}
\usetikzlibrary{shapes,arrows,positioning,calc,backgrounds}

\geometry{margin=2cm}

\title{\textbf{Parallélisation de l'Architecture DeepSeek} \\ 
       \large Guide de Transformation et Analyse Comparative}
\author{Architecture Système}
\date{\today}

\begin{document}
\maketitle

%------------------------------------------------
\section{Introduction}
Ce document présente les stratégies de parallélisation de l'architecture microservices DeepSeek existante et compare les performances entre l'architecture originale et l'architecture parallélisée.

%------------------------------------------------
\section{Architecture Originale vs Architecture Parallélisée}

\begin{table}[H]
\centering
\renewcommand{\arraystretch}{1.4}
\begin{tabular}{|p{5cm}|p{5cm}|p{5cm}|}
\hline
\textbf{Composant} & \textbf{Architecture Originale} & \textbf{Architecture Parallélisée} \\
\hline
\textbf{API Gateway} & 
1 instance monolithique &
4-8 replicas avec load balancing \\
\hline
\textbf{Moteur d'Inférence} &
Service unique &
32 replicas avec distribution GPU-aware \\
\hline
\textbf{Training Service} &
Exécution séquentielle &
Data Parallelism sur 8+ GPUs \\
\hline
\textbf{Base de Données} &
PostgreSQL single instance &
Cluster avec sharding et réplication \\
\hline
\textbf{Cache} &
Redis standalone &
Redis Cluster 6 nœuds \\
\hline
\textbf{Monitoring} &
Prometheus single server &
Prometheus federated + Thanos \\
\hline
\textbf{Communication} &
Synchronique directe &
Asynchrone + message queue \\
\hline
\end{tabular}
\caption{Comparaison architecture originale vs parallélisée}
\end{table}

%------------------------------------------------
\section{Stratégies de Parallélisation}

\subsection{1. Parallélisme des Données (Data Parallelism)}
\begin{figure}[H]
\centering
\begin{tikzpicture}[scale=0.7, every node/.style={font=\small}]
\node[draw, fill=blue!20, minimum width=3cm, minimum height=1cm] (data) at (0,0) {Dataset};
\node[draw, fill=green!20, minimum width=1cm, minimum height=1cm] (split1) at (2,1) {Shard 1};
\node[draw, fill=green!20, minimum width=1cm, minimum height=1cm] (split2) at (2,0) {Shard 2};
\node[draw, fill=green!20, minimum width=1cm, minimum height=1cm] (split3) at (2,-1) {Shard N};

\node[draw, fill=orange!30, minimum width=1.5cm, minimum height=1cm] (model1) at (4,1) {Model 1};
\node[draw, fill=orange!30, minimum width=1.5cm, minimum height=1cm] (model2) at (4,0) {Model 2};
\node[draw, fill=orange!30, minimum width=1.5cm, minimum height=1cm] (model3) at (4,-1) {Model N};

\node[draw, fill=red!20, minimum width=2cm, minimum height=1cm] (sync) at (6,0) {Sync Gradients};

\draw[->] (data) -- (split1);
\draw[->] (data) -- (split2);
\draw[->] (data) -- (split3);
\draw[->] (split1) -- (model1);
\draw[->] (split2) -- (model2);
\draw[->] (split3) -- (model3);
\draw[->] (model1) -- (sync);
\draw[->] (model2) -- (sync);
\draw[->] (model3) -- (sync);

\node[above=0.1cm of data] {\textbf{AVANT: Training séquentiel}};
\node[above=0.1cm of sync] {\textbf{APRÈS: Data Parallelism}};
\end{tikzpicture}
\caption{Transformation Data Parallelism pour le training}
\end{figure}

\textbf{Implémentation:}
\begin{verbatim}
# AVANT - Training séquentiel
python train.py --data dataset/ --gpu 0

# APRÈS - Data Parallelism
python -m torch.distributed.launch --nproc_per_node=8 \
    train.py --data dataset/ --batch_size 512
\end{verbatim}

\subsection{2. Parallélisme de Modèle (Model Parallelism)}
\begin{figure}[H]
\centering
\begin{tikzpicture}[scale=0.7, every node/.style={font=\small}]
\node[draw, fill=blue!20, minimum width=2cm, minimum height=3cm] (model) at (0,0) {Modèle Complet};
\node[draw, fill=green!20, minimum width=1.5cm, minimum height=1cm] (part1) at (3,1) {Couches 1-10};
\node[draw, fill=green!20, minimum width=1.5cm, minimum height=1cm] (part2) at (3,-1) {Couches 11-20};

\node[draw, fill=orange!30, minimum width=1.5cm, minimum height=1cm] (gpu1) at (5,1) {GPU 0};
\node[draw, fill=orange!30, minimum width=1.5cm, minimum height=1cm] (gpu2) at (5,-1) {GPU 1};

\draw[->] (model) -- node[above] {Partition} (part1);
\draw[->] (model) -- (part2);
\draw[->] (part1) -- (gpu1);
\draw[->] (part2) -- (gpu2);

\node[above=0.1cm of model] {\textbf{AVANT: Modèle sur 1 GPU}};
\node[above=0.1cm of gpu1] {\textbf{APRÈS: Model Parallelism}};
\end{tikzpicture}
\caption{Transformation Model Parallelism pour grands modèles}
\end{figure}

\textbf{Implémentation:}
\begin{verbatim}
# AVANT - Modèle sur GPU unique
model = LargeModel().cuda(0)

# APRÈS - Model Parallelism
model = LargeModel()
model = nn.DataParallel(model)  # ou model.split_across_gpus()
\end{verbatim}

\subsection{3. Parallélisme des Requêtes (Request Parallelism)}
\begin{figure}[H]
\centering
\begin{tikzpicture}[scale=0.7, every node/.style={font=\small}]
\node[draw, fill=blue!20, minimum width=1.5cm, minimum height=1cm] (gw) at (0,0) {Gateway};
\node[draw, fill=green!20, minimum width=1.5cm, minimum height=1cm] (inf) at (2,0) {Inference};

\node[draw, fill=blue!20, minimum width=1.5cm, minimum height=1cm] (gw2) at (5,1) {Gateway};
\node[draw, fill=green!20, minimum width=1cm, minimum height=1cm] (inf1) at (7,1.5) {Inf 1};
\node[draw, fill=green!20, minimum width=1cm, minimum height=1cm] (inf2) at (7,0.5) {Inf 2};
\node[draw, fill=green!20, minimum width=1cm, minimum height=1cm] (inf3) at (7,-0.5) {Inf N};

\draw[->] (gw) -- (inf);
\draw[->] (gw2) -- (inf1);
\draw[->] (gw2) -- (inf2);
\draw[->] (gw2) -- (inf3);

\node[above=0.1cm of gw] {\textbf{AVANT: Single instance}};
\node[above=0.1cm of gw2] {\textbf{APRÈS: Multiple replicas}};
\end{tikzpicture}
\caption{Transformation Request Parallelism pour l'inférence}
\end{figure}

%------------------------------------------------
\section{Guide de Migration Pas-à-Pas}

\subsection{Phase 1: Préparation de l'Infrastructure}
\begin{enumerate}
\item \textbf{Containerisation} des services existants
\item Mise en place de \textbf{Kubernetes} avec support GPU
\item Configuration du \textbf{Load Balancer} intelligent
\item Déploiement du \textbf{Service Mesh} (Istio)
\end{enumerate}

\subsection{Phase 2: Parallélisation des Services}
\begin{enumerate}
\item \textbf{API Gateway}: Passage de 1 à 4 replicas
\item \textbf{Inference Engine}: Scaling à 16 replicas GPU
\item \textbf{Training Service}: Implémentation Data Parallelism
\item \textbf{Base de données}: Migration vers cluster
\end{enumerate}

\subsection{Phase 3: Optimisations Avancées}
\begin{enumerate}
\item Implémentation du \textbf{Model Parallelism}
\item Configuration \textbf{auto-scaling} basé sur la charge GPU
\item Mise en place du \textbf{monitoring distribué}
\item Optimisations \textbf{réseau et stockage}
\end{enumerate}

%------------------------------------------------
\section{Comparaison des Performances}

\begin{table}[H]
\centering
\renewcommand{\arraystretch}{1.4}
\begin{tabular}{|l|c|c|c|}
\hline
\textbf{Métrique} & \textbf{Originale} & \textbf{Parallélisée} & \textbf{Amélioration} \\
\hline
Throughput (req/s) & 2,000 & 18,000 & 9.0x \\
Latence p95 (ms) & 450 & 85 & 5.3x \\
Temps Training (jours) & 7 & 1.5 & 4.7x \\
Utilisation GPU & 45\% & 92\% & 2.0x \\
Coût/Inference & 1.0 & 0.58 & 1.7x \\
Disponibilité & 99.5\% & 99.95\% & +0.45pp \\
RPO/RTO & 4h/2h & 15min/5min & 16x/24x \\
\hline
\end{tabular}
\caption{Comparaison quantitative des performances}
\end{table}

\subsection{Analyse des Résultats}

\begin{figure}[H]
\centering
\begin{tikzpicture}
\begin{axis}[
ybar,
enlargelimits=0.15,
legend style={at={(0.5,-0.15)},
anchor=north,legend columns=-1},
ylabel={Performance},
symbolic x coords={Throughput,Latence,Training,GPU Usage,Coût},
xtick=data,
nodes near coords,
nodes near coords align={vertical},
]
\addplot coordinates {(Throughput,2) (Latence,2) (Training,2) (GPU Usage,2) (Coût,2)};
\addplot coordinates {(Throughput,9) (Latence,5.3) (Training,4.7) (GPU Usage,2) (Coût,1.7)};
\legend{Originale,Parallélisée}
\end{axis}
\end{tikzpicture}
\caption{Amélioration des performances après parallélisation}
\end{figure}

%------------------------------------------------
\section{Impact sur la Stack Technologique}

\begin{table}[H]
\centering
\renewcommand{\arraystretch}{1.3}
\begin{tabular}{|p{4cm}|p{5cm}|p{5cm}|}
\hline
\textbf{Composant} & \textbf{Stack Originale} & \textbf{Stack Parallélisée} \\
\hline
\textbf{Orchestration} & Docker Compose & Kubernetes + Helm \\
\hline
\textbf{Training} & PyTorch standard & PyTorch DDP + DeepSpeed \\
\hline
\textbf{Inference} & Flask API & FastAPI + replicas multiples \\
\hline
\textbf{Base de données} & PostgreSQL single & PostgreSQL cluster + Patroni \\
\hline
\textbf{Cache} & Redis standalone & Redis Cluster \\
\hline
\textbf{Monitoring} & Prometheus standalone & Prometheus + Thanos + Grafana \\
\hline
\textbf{Réseau} & Load Balancer simple & Istio Service Mesh \\
\hline
\textbf{Stockage} & Disques locaux & Storage distribué (Ceph) \\
\hline
\end{tabular}
\caption{Évolution de la stack technologique}
\end{table}

%------------------------------------------------
\section{Considérations Opérationnelles}

\subsection{Complexité Accrue}
\begin{itemize}
\item \textbf{Monitoring distribué} nécessaire
\item \textbf{Débogage} plus complexe
\item \textbf{Gestion des dépendances} entre services
\item \textbf{Sécurité} renforcée requise
\end{itemize}

\subsection{Compétences Requises}
\begin{itemize}
\item Expertise en \textbf{Kubernetes} et \textbf{orchestration}
\item Connaissance des \textbf{architectures distribuées}
\item Maîtrise du \textbf{monitoring distribué}
\item Expérience en \textbf{optimisation GPU}
\end{itemize}

\subsection{Coûts}
\begin{itemize}
\item \textbf{Infrastructure} : Augmentation initiale de 30-40%
\item \textbf{Maintenance} : Hausse de 20-30%
\item \textbf{Formation} : Investissement en compétences
\item \textbf{ROI} : Retour sur 6-12 mois
\end{itemize}

%------------------------------------------------
\section{Recommandations de Migration}

\subsection{Pour les Petites Équipes}
\begin{enumerate}
\item Commencer par le \textbf{Request Parallelism}
\item Paralléliser l'\textbf{API Gateway} en premier
\item Ajouter des \textbf{replicas statiques}
\item Utiliser des \textbf{solutions managées} (DB, cache)
\end{enumerate}

\subsection{Pour les Grandes Équipes}
\begin{enumerate}
\item Implémenter le \textbf{Data Parallelism} complet
\item Déployer un \textbf{Service Mesh}
\item Mettre en place l'\textbf{auto-scaling}
\item Construire une plateforme \textbf{self-service}
\end{enumerate}

%------------------------------------------------
\section{Conclusion}

\subsection{Bénéfices de la Parallélisation}
\begin{itemize}
\item \textbf{Performance} : 4-9x amélioration selon les métriques
\item \textbf{Scalabilité} : Support de croissance exponentielle
\item \textbf{Disponibilité} : Meilleure résilience aux pannes
\item \textbf{Efficacité} : Meilleure utilisation des ressources
\end{itemize}

\subsection{Recommandations Finales}
\begin{enumerate}
\item \textbf{Commencer progressivement} avec les services critiques
\item \textbf{Mesurer continuellement} les performances
\item \textbf{Investir en formation} de l'équipe
\item \textbf{Planifier la migration} sur 3-6 mois
\item \textbf{Maintenir la compatibilité} pendant la transition
\end{enumerate}

La parallélisation de l'architecture DeepSeek offre des gains significatifs en performance et scalabilité, mais nécessite une planification soigneuse et des compétences spécialisées. L'approche incrémentale est recommandée pour minimiser les risques.

\end{document}