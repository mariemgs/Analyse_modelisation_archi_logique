\documentclass[12pt,a4paper]{article}
\usepackage[utf8]{inputenc}
\usepackage[french]{babel}
\usepackage{geometry}
\usepackage{tikz}
\usepackage{xcolor}
\usepackage{amsmath,amsfonts,amssymb}
\usepackage{graphicx,float,hyperref,array}
\usetikzlibrary{shapes,arrows,positioning,calc,backgrounds}

\geometry{margin=2cm}

\title{\textbf{Architecture Microservices de DeepSeek} \\ 
       \large Modélisation UML d'une Plateforme d'Intelligence Artificielle}
\author{Architecture Système}
\date{\today}

\begin{document}
\maketitle
\tableofcontents
\newpage

%------------------------------------------------
\section{Introduction}
Cette document présente l'architecture microservices hypothétique de DeepSeek, une plateforme d'intelligence artificielle avancée. L'architecture est conçue pour supporter la scalabilité, la haute disponibilité et la performance nécessaires pour les workloads d'IA modernes.

%------------------------------------------------
\section{Vue d'ensemble de l'Architecture}
\begin{figure}[H]
\centering
\begin{tikzpicture}[scale=0.85, every node/.style={font=\small}]
% Main Layers
\draw[fill=blue!15, rounded corners] (0,8) rectangle (12,9.5);
\node at (6,8.75) {\textbf{API Gateway \& Load Balancer}};

\draw[fill=green!15, rounded corners] (0,6.5) rectangle (12,7.5);
\node at (6,7) {\textbf{Authentication \& User Management Layer}};

\draw[fill=orange!15, rounded corners] (0,4.5) rectangle (12,5.5);
\node at (6,5) {\textbf{Core AI Services: Training, Inference, Fine-Tuning}};

\draw[fill=purple!15, rounded corners] (0,2.5) rectangle (12,3.5);
\node at (6,3) {\textbf{Data Layer: Pipelines, Vector DB, Cache}};

\draw[fill=red!15, rounded corners] (0,0.5) rectangle (12,1.5);
\node at (6,1) {\textbf{Infrastructure: Monitoring, Logging, Discovery}};
\end{tikzpicture}
\caption{Vue d'ensemble de l'architecture microservices DeepSeek}
\vspace{0.3cm}
\begin{center}
\colorbox{gray!10}{
\begin{minipage}{0.9\textwidth}
\textbf{Légende :} 
\textcolor{blue!60}{Bleu} — API \& Gateway \hspace{0.5cm}
\textcolor{green!60}{Vert} — Authentification \& Gestion Utilisateur \hspace{0.5cm}
\textcolor{orange!80!black}{Orange} — Services IA Cœur \hspace{0.5cm}
\textcolor{purple!70!black}{Violet} — Données \hspace{0.5cm}
\textcolor{red!70!black}{Rouge} — Infrastructure
\end{minipage}}
\end{center}
\end{figure}

\newpage
%------------------------------------------------
\section{Diagramme de Déploiement}
\begin{figure}[H]
\centering
\begin{tikzpicture}[scale=0.8, every node/.style={font=\scriptsize}]
\draw[thick, rounded corners] (0,0) rectangle (14,10);
\node[font=\bfseries] at (7,9.5) {Cluster Kubernetes DeepSeek};

\foreach \y/\name/\color in {8.2/gateway/blue,6.2/ai-services/orange,4.2/data-services/purple,2.2/infrastructure/red}{
    \draw[fill=\color!10, rounded corners] (0.5,\y) rectangle (13.5,\y+1);
    \node[font=\bfseries] at (7,\y+0.5) {Namespace : \name};
}
\end{tikzpicture}
\caption{Diagramme de déploiement Kubernetes simplifié et lisible}
\vspace{0.3cm}
\begin{center}
\colorbox{gray!10}{
\begin{minipage}{0.9\textwidth}
\textbf{Légende :}
\textcolor{blue!70}{Gateway} — Interface d’entrée \hspace{0.5cm}
\textcolor{orange!80!black}{AI Services} — Traitement IA \hspace{0.5cm}
\textcolor{purple!70!black}{Data Services} — Données et stockage \hspace{0.5cm}
\textcolor{red!70!black}{Infrastructure} — Support technique
\end{minipage}}
\end{center}
\end{figure}

\newpage
%------------------------------------------------
\section{Diagramme de Composants UML Amélioré}
\begin{figure}[H]
\centering
\begin{tikzpicture}[
component/.style={rectangle, draw=black!60, fill=teal!20, rounded corners,
minimum width=4.2cm, minimum height=2cm, align=center, font=\small},
dependency/.style={->, dashed, thick, teal!70}]
\node[component] (api) at (0,10) {\textbf{API Gateway}\\+ routeRequest()\\+ authenticate()};
\node[component] (auth) at (5,10) {\textbf{Auth Service}\\+ validateToken()\\+ login()};
\node[component] (mesh) at (2.5,8) {\textbf{Service Mesh}\\(sécurité, observabilité)};
\node[component] (inf) at (2.5,5.5) {\textbf{Inference Engine}\\+ processInference()};
\node[component] (train) at (-2,3) {\textbf{Training Service}\\+ startTraining()};
\node[component] (vector) at (5,3) {\textbf{Vector DB}\\+ storeVector()\\+ searchSimilar()};
\node[component] (monitor) at (1.5,0.5) {\textbf{Monitoring}\\(Prometheus/Grafana)};
\draw[dependency] (api) -- (auth);
\draw[dependency] (auth) -- (mesh);
\draw[dependency] (mesh) -- (inf);
\draw[dependency] (inf) -- (train);
\draw[dependency] (train) -- (vector);
\draw[dependency] (vector) -- (monitor);
\end{tikzpicture}
\caption{Diagramme de composants UML - Architecture DeepSeek}
\vspace{0.3cm}
\begin{center}
\colorbox{gray!10}{
\begin{minipage}{0.9\textwidth}
\textbf{Légende :}
Les flèches bleues indiquent les dépendances fonctionnelles.  
Chaque rectangle représente un microservice indépendant déployé dans Kubernetes.
\end{minipage}}
\end{center}
\end{figure}

\newpage
%------------------------------------------------
\section{Diagramme de Séquence – Traitement d'une Requête d'Inférence}
\begin{figure}[H]
\centering
\begin{tikzpicture}[scale=0.8, every node/.style={font=\small}]
\node (client) at (0,8) {Client};
\node (gateway) at (2.5,8) {API Gateway};
\node (auth) at (5,8) {Auth Service};
\node (inf) at (7.5,8) {Inference Engine};
\node (cache) at (10,8) {Cache};
\node (model) at (12.5,8) {Model Engine};

\foreach \x in {client,gateway,auth,inf,cache,model}
  \draw[dashed] (\x) -- ++(0,-7);

\draw[->, thick] (0,7.5) -- (2.5,7.3) node[midway, above] {HTTP Request};
\draw[->] (2.5,7) -- (5,6.8) node[midway, above] {Validate Token};
\draw[->] (5,6.5) -- (2.5,6.3) node[midway, above] {Auth OK};
\draw[->] (2.5,6) -- (7.5,5.8) node[midway, above] {Forward Request};
\draw[->] (7.5,5.5) -- (10,5.3) node[midway, above] {Cache Check};
\draw[->] (10,5) -- (7.5,4.8) node[midway, above] {Cache Miss};
\draw[->] (7.5,4.5) -- (12.5,4.3) node[midway, above] {Run Model};
\draw[->] (12.5,4) -- (7.5,3.8) node[midway, above] {Result};
\draw[->] (7.5,3.5) -- (10,3.3) node[midway, above] {Store Cache};
\draw[->] (7.5,3) -- (2.5,2.8) node[midway, above] {Response};
\draw[->] (2.5,2.5) -- (0,2.3) node[midway, above] {HTTP Response};
\end{tikzpicture}
\caption{Diagramme de séquence : flux de traitement d'inférence}
\vspace{0.3cm}
\begin{center}
\colorbox{gray!10}{
\begin{minipage}{0.9\textwidth}
\textbf{Légende :}  
Ce diagramme illustre le parcours complet d’une requête d’inférence :  
le client envoie une requête HTTP, validée par l’API Gateway et Auth Service,  
puis traitée par le moteur d’inférence avant d’être mise en cache et retournée.
\end{minipage}}
\end{center}
\end{figure}

\newpage
%------------------------------------------------
\section{Diagramme d'Activité – Pipeline de Training}
\begin{figure}[H]
\centering
\begin{tikzpicture}[
start/.style={ellipse, draw, fill=green!25, minimum width=1.8cm},
activity/.style={rectangle, draw, fill=blue!20, minimum width=2.8cm, rounded corners},
decision/.style={diamond, draw, fill=yellow!25, aspect=2},
end/.style={ellipse, draw, fill=red!25, minimum width=1.8cm}]
\node[start] (start) at (0,8) {Début};
\node[activity] (val) at (0,6.5) {Valider Dataset};
\node[activity] (prep) at (0,5) {Préparer Données};
\node[activity] (init) at (0,3.5) {Initialiser Training};
\node[activity] (train) at (0,2) {Entraîner Modèle};
\node[decision] (conv) at (3.5,2) {Convergé?};
\node[activity] (save) at (3.5,0.5) {Sauvegarder Checkpoint};
\node[activity] (eval) at (7,2) {Évaluer Modèle};
\node[activity] (dep) at (7,0.5) {Déployer Modèle};
\node[end] (end) at (7,3.5) {Fin};
\draw[->] (start) -- (val) -- (prep) -- (init) -- (train) -- (conv);
\draw[->] (conv) -- node[right]{Non} (save);
\draw[->] (save) |- (train);
\draw[->] (conv) -- node[above]{Oui} (eval);
\draw[->] (eval) -- (dep) -- (end);
\end{tikzpicture}
\caption{Diagramme d’activité : pipeline d’entraînement des modèles}
\vspace{0.3cm}
\begin{center}
\colorbox{gray!10}{
\begin{minipage}{0.9\textwidth}
\textbf{Légende :}
Le pipeline d'entraînement suit une boucle continue entre l'entraînement et l’évaluation jusqu’à convergence.  
Les checkpoints sont enregistrés pour assurer la reprise et le suivi de performance.
\end{minipage}}
\end{center}
\end{figure}

\newpage
%------------------------------------------------
\section{Patterns Architecturaux Utilisés}
\subsection{Microservices}
L'architecture est basée sur le pattern microservices pour assurer:
\begin{itemize}
    \item \textbf{Scalabilité}: Chaque service peut être scalé indépendamment
    \item \textbf{Résilience}: La panne d'un service n'affecte pas les autres
    \item \textbf{Déployabilité}: Déploiements indépendants et fréquents
    \item \textbf{Technology Diversity}: Choix de technologies adaptées par service
\end{itemize}

\subsection{API Gateway}
Centralise la gestion des requêtes avec:
\begin{itemize}
    \item Authentification et autorisation
    \item Rate limiting et throttling
    \item Load balancing
    \item Monitoring et logging
\end{itemize}

\subsection{CQRS et Event Sourcing}
Séparation des opérations de lecture et d'écriture pour optimiser les performances.

\subsection{Circuit Breaker}
Protection contre les cascades de pannes avec des mécanismes de fallback.

\newpage
%------------------------------------------------
\section{Technologies Utilisées}
\begin{table}[H]
\centering
\renewcommand{\arraystretch}{1.3}
\begin{tabular}{|l|l|l|}
\hline
\textbf{Couche} & \textbf{Technologie} & \textbf{Usage} \\
\hline
Orchestration & Kubernetes & Déploiement et gestion des conteneurs \\
API Gateway & Kong / Envoy & Routage et sécurité \\
Services & Python / Go & Logique métier \\
Base de données & PostgreSQL & Données relationnelles \\
Cache & Redis & Cache distribué \\
Message Queue & Apache Kafka & Communication asynchrone \\
Vector DB & Milvus / Pinecone & Stockage et recherche vectorielle \\
Monitoring & Prometheus / Grafana & Surveillance et métriques \\
Logging & ELK Stack & Agrégation et analyse des logs \\
Stockage & MinIO / S3 & Stockage d’objets \\
\hline
\end{tabular}
\caption{Stack technologique}
\end{table}

\newpage
%------------------------------------------------
\section{Scalabilité et Performance}
\subsection{Stratégies de Scalabilité}
\begin{itemize}
    \item \textbf{Horizontal Scaling}: Auto-scaling basé sur la charge CPU/mémoire
    \item \textbf{GPU Scaling}: Allocation dynamique des ressources GPU
    \item \textbf{Database Sharding}: Partitionnement des données par tenant
    \item \textbf{CDN}: Distribution géographique du contenu statique
\end{itemize}

\subsection{Optimisations Performance}
\begin{itemize}
    \item Cache multi-niveaux (L1: In-memory, L2: Redis)
    \item Pré-chargement des modèles fréquemment utilisés
    \item Batch processing pour les requêtes d'inférence
    \item Compression des modèles et quantization
\end{itemize}

\newpage
%------------------------------------------------
\section{Sécurité}
\subsection{Authentification et Autorisation}
\begin{itemize}
    \item OAuth 2.0 / JWT pour l'authentification
    \item RBAC (Role-Based Access Control)
    \item API Keys pour les intégrations
    \item mTLS entre microservices
\end{itemize}

\subsection{Sécurité des Données}
\begin{itemize}
    \item Chiffrement end-to-end
    \item Vault pour la gestion des secrets
    \item Network policies Kubernetes
    \item Audit logging complet
\end{itemize}

\newpage
%------------------------------------------------
\section{Monitoring et Observabilité}
\subsection{Métriques Clés}
\begin{itemize}
    \item \textbf{Business}: Requêtes/sec, latence p95, taux d'erreur
    \item \textbf{Technique}: CPU, mémoire, GPU utilisation
    \item \textbf{Modèles}: Accuracy, drift detection, performance
\end{itemize}

\subsection{Alerting}
Configuration d'alertes pour:
\begin{itemize}
    \item Dégradation de performance
    \item Erreurs critiques
    \item Ressources insuffisantes
    \item Anomalies de sécurité
\end{itemize}

%------------------------------------------------
\section{Conclusion}
Cette architecture microservices pour DeepSeek offre:

\begin{itemize}
    \item \textbf{Haute disponibilité}: 99.9\% uptime grâce à la redondance
    \item \textbf{Scalabilité élastique}: Auto-scaling basé sur la demande
    \item \textbf{Performance optimale}: Cache intelligent et GPU management
    \item \textbf{Sécurité robuste}: Authentification, chiffrement, audit
    \item \textbf{Maintenabilité}: Code modulaire et déploiements indépendants
\end{itemize}

L'architecture peut évoluer pour intégrer de nouveaux modèles d'IA, supporter plus d'utilisateurs et s'adapter aux besoins futurs de la plateforme.

\end{document}
